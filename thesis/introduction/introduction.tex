
\chapter{Introduction}

\section{Motivation and Objectives}
Portfolio optimisation is not a new concept. 



In this project, we attempt to use SMC to solve the portfolio optimisation problem. The fundamental goal of portfolio theory is to optimally allocate resources to different investible assets. The meaning of optimality may be different to the investors. Some investors may have tax considerations; others may have different legal restrictions on investments or holiding periods. Two common objectives, often contradicting, are the investment risk and return. We adopt here the Markowitz's mean-variance efficient portfolio which allow one to make the portfolio allocation by considering the trade-off between expected risk and expected return. 

\section{Contributions}

Contributions here.


\section{Tecnical Approach}


\section{Thesis organsiation}
The subsequent chapters of this thesis are organised as follows:
\begin{enumerate}
\item Chapter 2 provides an overview on the Monte Carlo methods. It begins with a brief introduction to basic methods such as perfect Monte Carlo sampling, rejection sampling, importance sampling. It then details two common Markov Chain Monte Carlo (MCMC) techniques, namely Metropolis-Hastings and Gibbs Sampling. Lastly, it introduces the SMC technique used in this thesis. 
\item Chapter 3 presents the probl
\item Chapter 4 the formulation of the experiments in using SMC to discover the optimal policies for the given portfolio. It begins with the problem formulation.

\item Chapter 5 concludes the thesis by evaluating the degree to which the hypothesis has been justfied and outlines potential work for the future.
\end{enumerate}

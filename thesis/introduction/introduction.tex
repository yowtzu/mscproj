\chapter{Introduction}
Resource allocation is a common challenge for every investor. In the investment decision making process, investors decide how much resources are allocated to different investable assets to form a portfolio with the aim to optimise the performance of the overall portfolio is better off than any other according to some criterion. The criterion can be different to the investors. (Some investors may have different considerations such as tax considerations and legal restriction on investment assets or holding periods.)

Two common objectives, often contradicting, are the financial return and the investment risk. The Markowitz's modern portfolio theory \cite{HM58} proposes a portfolio selection framework. In Markowitz's model, it is assumed that investor attempt to maximize a portfolio's return and minimize the risk (measured by the variance of the portfolio return. Based on this criteria, the set of non-dominance portfolio is known as the \emph{efficient porfolios}. Using variance as a ris measure has its limitation. Variance is a symmetric measure; an out performing asset (than the expected return) is deemed to as risky as an under perfoming one. Many alternatve risk measurements have been proposed, e.g. Sortino ratio, Conditional Value at Risk (CVaR), etc.. Refer \cite{X} for details.

In the original Marowitz model, the investment decision problem is viewed and solved as a single time-step problem. In practice, investment often span across multi time-step period and adjustments may be made to the allocation periodically to achive better performance as needed. This problem is much more difficult to deal with because it is time inconsistent in the sense that an investment strategy that is optimal over the whole period may not be the optimal one over a sub-interval of the period.This violates the Bellman's Priciple of Optimality \cite{X}. Consequently, dynamic programming approach is not applicable here. 

\section{Technical Approach}
Traditionally, multi-period mean-variance portfolio optimization have been explired in an analytical fashion, adopting necessary assumption as necessary (see review at Section \ref{}) . This seems rather restrictive; there are many instances where numerical method has been used to derive an approximate or even more effective solution to the problem in question. For example, Monte Carlo technique is used to do integral, evolutionary techniques applied in engineering domains, etc..

Our approach to the problem in this thesis is a radical one. We view a portfolio optimisation as a \emph{stochastic} control problem. We adopt the Bayesian view and treat tehse parameters as random variables. The objective is to find the sequence of control parameters that minimize the control objective defined in terms of portfolio return and financial risk. We investigate the potential of using SMCs as the means of determining the optimal strategy, or at least excellent, strategies for multi-period mean-variance portfolio optimisation problem. The main reason of choosing SMCs is its ability to carry out \emph{sequential} update on the posterior distribution over time fit well with parameter inference in stochastic process. Moreoever, these techniques have achieved significant success in their applications on many domains. Of course, other heuristic search tecniques are also potentially applicable.

To investigate this approach, we considered a simplied a market model with two assets: one risky asset with its price modelled as Brownian motion with drift, and one zero interest risk-free asset (constant) which has an analytical solution. Using SMCs, we search for the optimal strategy (the set of control parameters at each time point) that optimise against the optimisation criteria (a.k.a. reward) in terms of expected return over the investment period, and minimize the financial risk (variance of the return) using SMC techniques. The results are then evaluated against with the analytic solution described in \cite{}.

\section{Thesis hypothesis}
Formally, the hypothesis of the thesis is stated as follows:
\begin{quote}
Sequential Monte Carlo (SMCs) have the potential to be an effective means of searching the optimal strategy for portfolio problem.
\end{quote}
We attempt to examine this hypothesis from two different perspectives:
\begin{enumerate}
\item Exploring the potential of EAs in searching the optimal strategy from mean-variance criteria.
\item Exploring the potential of GPUs to improve the optimisation performance in terms of speed and accuracy.
\end{enumerate}
Given the time frame, we fully understand it is impossible to evaluate our approach on full scale strategy. The aim is to establish the plausibility or, at the very least, a greater understanding of the strengths and weaknesses of the above approach.

\section{Thesis organisation}
The subsequent chapters of this thesis are organised as follows:
\begin{enumerate}
\item Chapter 2 provides a review on the Monte Carlo methods. It begins with a brief introduction to basic methods such as perfect Monte Carlo sampling, rejection sampling, importance sampling. It then details two common Markov Chain Monte Carlo (MCMC) techniques, namely Metropolis-Hastings and Gibbs Sampling. Lastly, it introduces the SMC technique used in this thesis. 
\item Chapter 3 provides a review on GPU architecture.
\item Chapter 4 the formulation of the experiments in using SMC to discover the optimal policies for the given portfolio. It begins with the problem formulation.
\item Chapter 5 concludes the thesis by evaluating the degree to which the hypothesis has been justfied and outlines potential work for the future.
\end{enumerate}

%In this project, we attempt to use SMC to solve the portfolio optimisation problem. The fundamental goal of portfolio theory is to optimally allocate resources to different investible assets. The meaning of optimality may be different to the investors. Some investors may have tax considerations; others may have different legal restrictions on investments or holiding periods. Two common objectives, often contradicting, are the investment risk and return. We adopt here the Markowitz's mean-variance efficient portfolio which allow one to make the portfolio allocation by considering the trade-off between expected risk and expected return. 


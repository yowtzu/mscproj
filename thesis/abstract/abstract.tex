
% Thesis Abstract -----------------------------------------------------


%\begin{abstractslong}    %uncommenting this line, gives a different abstract heading
\begin{abstracts}        %this creates the heading for the abstract page
  Constructing an optimal portfolio is the ultimate goal of every investment manager; but the optimality criteria may be very different to each of them. For index tracker fund manager, the main objective of portfolio management is to track and replicate the exposure of a benchmark index.

In this study, we view this as a stochastic control problem. We adopt the Bayesian view and treat the investment decision controls as random variables. The objective is to search for the sequence of control parameters that results in a portfolio that tracks a benchmark index in an optimal way. We investigate here the potential of using Sequential Monte Carlo (SMC) techniques as the means of determining such strategy. We examine the feasibility of this approach using two examples. The first example is a toy example with the target reference set to be oscillating oscillating wave. In the second example, a real world index --- the German's DAX index level is used as the target reference. In both cases, the approach looks very promising. Lastly, we demonstrate how the Model Preditive Control (MPC) concept can be incorporated to improve tracking performance.

  The thesis concludes with an evaluation on the work done, the extent
  to which the work justify the thesis hypothesis and some possible
  directions on how SMC techniques can be applied to address
  a wider range of relevant problems on the domain of concern.
\end{abstracts}
%\end{abstractlongs}


% ----------------------------------------------------------------------


%%% Local Variables: 
%%% mode: latex
%%% TeX-master: "../thesis"
%%% End: 


% Thesis Abstract -----------------------------------------------------


%\begin{abstractslong}    %uncommenting this line, gives a different abstract heading
\begin{abstracts}        %this creates the heading for the abstract page
  Optimal portfolio is the ultimate goal of every investment manager; but the optimality criterion can be very different to each of them. For index tracker fund manager, the main objective of portfolio management is to track and replicate the exposure of a benchmark index.

In this study, we view this as a stochastic control problem. We adopt the Bayesian view and treat these parameters as random variables. The objective is to find the sequence of control parameters that results in a portfolio that tracks a benchmark index in an optimal way. We investagete here the potential of using Sequential Monteo Carlo (SMC) as the means of determining such strategy. We examine the feasibility of this approach using two examples. The first example is a toy example with the target reference set to be oscillating  oscillating wave. In the second example, the DAX index level is used as the target reference signal. In both cases, the approach looks very promising.

  The thesis concludes with an evaluation on the work done, the extent
  to which the work justify the thesis hypothesis and some possible
  directions on how SMC can be applied to address
  a wider range of relevant problems on the domain of concern.
\end{abstracts}
%\end{abstractlongs}


% ----------------------------------------------------------------------


%%% Local Variables: 
%%% mode: latex
%%% TeX-master: "../thesis"
%%% End: 

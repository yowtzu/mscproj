\chapter{Evaluation and conclusions}
\graphicspath{{Chapter5/figures/}}
\label{EvaluationAndConclusion}
The work reported in previous chapters provides evidence to support
the thesis hypothesis stated in
Section, namely:
\begin{quote}
  Evolutionary algorithms~(EAs) have the potential to be an effective
  means of determining the security policies that suit dynamic
  challenging environments.
\end{quote}
This chapter reviews the work that has been done, evaluates the extent
to which they justify the thesis hypothesis and concludes the thesis
by addressing the directions for future work.

\section{Evaluation}
\label{Evaluation}
In previous three chapters, We have detailed several experimentations
that serves to support the thesis hypothesis from three different
strands of research. We explored the potential of EAs in inferring
optimal security policies, dynamically updating security policies with
new decision examples and searching for policies with optimal trade
offs between objectives using simulation runs. This section summarises
the work completed in each strand of research and outlines the
contributions and novelty of the work presented in this thesis.

\subsection{Static policy inference}
\label{Evaluation.StaticSecurityPolicyInference}
Current security policy is often developed in a top-down
approach. High-level security goals are first determined, after which
they undergo a series of refinement processes to obtain the low-level
executable rules. Although some work has been done in applying machine
learning techniques to aid the policy refinement process, there is no
previous work to my knowledge in the application of EAs or machine
learning techniques in inferring security policies.

Chapter  details the experiments in
using EAs to infer security policies from decision examples. Here EAs
is used as a tool to generalise a set of low-level examples to a set
of high-level rules. Various simple security policies have been
attempted and inferred successfully. These include the traditional MLS
Bell-LaPadula policy model, the budgetised MLS policy model and the
Fuzzy MLS policy model. Two different EAs, namely GP and GE are
used. In all cases, the results show that a minimal amount of design
effort and domain knowledge are required to infer the reference policy
or a close approximator of it. The only requirements are to have a
good fitness function and training examples that form a good
representation of the target policy.

The last part of the chapter presents how other machine techniques can
be incorporated into the policy inference framework created. Fuzzy set
concept is used as an example here. Multiple policies are learnt
independently; each of which focuses on inferring a fuzzy rule for a
particular class of decisions~(fuzzification). The ultimate output
policy, which is an ensemble of all these policies, is formed using a
weighted voting mechanism~(defuzzification). Various experiments have
been carried out to examine different fuzzification and
defuzzification techniques. The results show that these approaches can
consistently infer policies that closely match with the original
reference models used.

\subsection{Dynamic policy inference}
\label{Evaluation.DynamicSecurityPolicyInference}
There will inevitably be times when unseen circumstances demand a
decision during operation. In some cases the default automated
response may be imperative; in other cases this may be
ill-advised. Manual decisions made to override the default one
essentially define a new policy. Furthermore, even if the optimal
security policy can be developed or inferred automatically, it would
eventually become suboptimal due to the changes in either the
operational environment or security requirements, or both. Therefore, a
security policy has to be able to continually change and be updated to
suit the operational needs to maintain its optimality.


on dynamic security policy inference. As there is no dynamic security
policy model available and therefore no decision example is available
for us to work with, we designed a dynamic security policy model. This
model is used to generate time varying decision examples for training
and evaluation purposes.

To infer this dynamic security policy model, two novel dynamic
learning frameworks based upon MOEAs are designed: one that based on
Fan's intuition~\cite{FW04} and DOO. In DOO, an $n$-objective
optimisation problem is treated as a $2n$-objective optimisation
problem by adding an opposing objective for each of the original
objectives. With such a setting, DOO is able to maintain the diversity
among the individuals in the population whilst optimising the intended
objectives. This diversity can aid in preventing the population from
premature convergence and allows the concept drift in the policy to be
continually relearnt. The results show that these frameworks are very
promising. Reasonably good approximators to the model are able to be
inferred from the examples using these frameworks.

\subsection{Mission-specific policy discovery}
\label{Evaluation.MissionSpecificPolicyDiscovery}
Chapter introduces the notion of
mission-specific policy discovery. EAs are used to search for the
security policies that can provide the optimal, or at least excellent,
tradeoffs among security objectives for a specific mission. Here, EAs
serve as an optimisation tool to synthesise the optimal policies, in
terms of achieving the mission as well as security objectives without
violating the constraints given.

We demonstrate here how simulation can be used to obtain the fitnesses
of the policy candidates that are used to guide the policy search. To
evaluate the fitness of an individual~(policy) for a mission, the
policy is first plugged into a simulated mission, then the simulated
mission is executed and the outcome of it is measured. This is very
different from the practice of fitting a policy a priori without the
details of the specific mission being taken into account. This concept
of ``mission-specific policy'' is entirely novel.

Various EA baed techniques are used here to discover the optimal
policies. These include GP/MOGP and DOO. In all cases, the results
show that these techniques are able to discover the set of policies
that are optimal for the mission of concern. By using MOGP (and MOEAs
in general), tradeoffs between a variety of criteria can be
explored. Such information can be valuable to policy makers to select
and apply the optimal policy that best fits the current operation.

\subsection{Thesis contributions}
In summary, we demonstrate how:
\begin{itemize}
\item EAs can be used to infer static security policies from a set of
  decision examples. Three different ways of representing security
  policies and two different EAs are investigated. The results show
  that this idea is feasible.
\item the fuzzy set concept can be integrated into the policy
  inference framework to improve the policy inference performance. The
  idea is sufficiently generic to be applied to other classification
  problems, provided that there is a partial ordering among the
  classes.
\item multi-objective evolutionary algorithms~(MOEAs) can be used to
  infer dynamic security policies from a set of decision examples. Two
  novel dynamic learning frameworks based upon MOEAs are developed:
  one that is based on Fan's intuition and DOO. Both of them can be
  used as general dynamic classification algorithms.
\item an ensemble policy model can be constructed from multiple models
  in a single EA run to achieve better performance. The improvement is
  especially significant in the DOO setting.
\item MOEAs can be used to infer a set of Pareto optimal policies that
  fit a specific mission~(or at least a specific family of missions).
\item simulation runs can be used in place of a set of decision
  examples to provide feedback in evaluating the fitness of a policy
  with respect to the specified high-level objectives.
\item MOEAs can be used as a decision making tool where tradeoffs
  between objectives exists. The Pareto front of the security policies
  discovered using MOEAs can reveal useful information about the
  relationship among different objectives, which may be difficult to
  obtain otherwise. Such information provides useful insight for
  policy makers to select and apply the optimal policy that fits the
  needs of current operational environment on a case-by-case basis.
\end{itemize}

\section{Envisaged future work}
Having discussed the contributions of the thesis, we now outline
numerous possible directions for future work that have been identified
during the course of this research.

\subsection{Policy fusion}
In dynamic coalitions, parties with different policies can come
together to collaborate. Prior to the formation of dynamic coalitions,
each party may have its own security policy. An interesting step
forward would be to investigate how well EAs could be used to combine
these security policies together. One possible way is to generate
decision examples from both existing policies and use these examples
as the training input for the policy inference framework. MOEAs can
also be used to discover the Pareto optimal set of policy candidates,
which are then chosen depending on the security requirements. However
there are still issues that require further investigation. These
include:
\begin{itemize}
\item Understanding how to deal with policies that consists of
  different sets of decision-making factors, which may be measured
  using different scales.
\item Understanding what the implicit priorities that EAs have
  assigned to the conflicting rules are, what the factors that
  influence the priorities are and how to control these priorities,
  etc.
\end{itemize}

\subsection{The robustness of a security policy}
The framework proposed in this thesis has been shown to be effective
in dynamically inferring the optimal policy. However, the optimality
of a policy is not always the only factor of concern; the robustness
in performance of a security policy in different environments may be
equally important. This is especially so in a pervasive operating
environment where the deployment of a new policy can be a difficult or
expensive process. To incorporate this factor into the proposed
framework, a way to quantify the robustness in performance of a
security policy is required.
%  One simple way to quantify the
%  robustness in performance of a policy is defined as follows:
%  \begin{enumerate}
%  \item Split the decision examples in the training set into multiple
%    subsets, possibly based on the time when the decision is made.
%  \item Measure the performance of the security policy on the examples
%    in each subset.
%  \item The robustness in performance of the security policy can be
%    estimated with the variance of the performance of the security
%    policy over all the subsets. More robust security policy has
%    smaller variance.
%  \end{enumerate}

This measure also provides a way to determine the invariant part of
the optimal policies for different operational environments of
concern. The determination of this invariant part is doubly useful:
Firstly, it can serve as a template or testing target in the policy
development process. Secondly, it can help to protect the security
policy inference framework from poisoning attack, which attempts to
mislead the inference process in the favour of the attacker by the
injection of specially crafted decision examples.

\subsection{Scalability with the training set size}
Scalability is a subtle issue. We have addressed some aspects of this
issue. For example, we have shown the method scales well with the size
of the training set. In the experiments presented in
Chapter  we have increased the
size of the training set from $100$ examples to $1000$ examples and
the results still remain consistent. Obviously, the fitness evaluation
time would increase; $1000$ examples take ten times longer than $100$
examples to evaluate. This is unlike to be an issue in practice as the
fitness evaluation of each individual can be executed in parallel if
necessary. In Chapterwe have
shown that DOO is able to evolve and update policies with decision
examples in an incremental manner. However, there are still some
issues remaining with these frameworks that need to be
investigated. This includes searching for appropriate techniques to
sample old decision examples and examining the generality of the DOO
framework.

\subsection{More complex security policies}
The security policies used in this thesis are rather simple. This can
be potentially an issue. However, note that these policies are either
real-world policies or proposals from major research institutes for
real world use. They are simple, but by no means ``toy''
policies. Ultimately, we should strive for simple policies wherever is
possible, but at the same time, we should also need to acknowledge
that MANET policies may need legitimately to be much more
complicated. To cope with complexity, instead of attempting to extract
and discover the policies as a whole, we could simply target the areas
that we need help. Humans produce security policies sequentially too,
i.e., they consider in turn authentication policy, file access
control, audit policy, etc. In practice, it is also often that there
are some rules of thumbs and constraints that are dictated from on
high. We do not need to extract these bits of a policy. Yet, there is
still much to answer here, for example:
\begin{itemize}
\item Can EAs be used to evolve more complex policies or policies of
  other types, e.g., obligation policies? If not, how can we divide
  the security policies into smaller components in a systematic
  manner?
\item How to incorporate the constraints imposed from on high into
  the policy inference framework to form a continuous learning loop
  in an efficient manner? Should we take such constraints into
  consideration in the evolution process? If so, how?
\end{itemize}

\subsection{More complex scenarios}
The scenario used in Chapter is
relatively simple. It has only one type of agent in each team and one
type of information. A real test of this approach would be to embed it
within a more realistic simulated scenario, with more sophisticated
information types, and realistic consequence models. Note that the
simulated scenario may be much more complex but we are really
interested in some of the measurable properties, which may only be
few. For example, how many properties would an operational commander
be interested in feasibly ready trading off? The techniques proposed
here should be able to scale well with it.

\section{Closing remarks}
\label{ClosingRemark}
The work reported in this thesis demonstrates a considerable degree of
originality supported by extensive experimentation. The case studies
are necessarily limited given the limited amount of time frame.  However, the results
 demonstrate that portfolio optimisation
approaches using Sequential Monte Carlo techniques have very considerable promise. Everyone accepts that portfolio optimisation is difficult, and things are to worsen as the marketfinancial become more complex environments with increasing sophistication and
subtlety of decision-making process. We recommend these approaches to
the research community for further investigation.


%%% ----------------------------------------------------------------------

% ------------------------------------------------------------------------

%%% Local Variables: 
%%% mode: latex

%%% TeX-master: "../thesis"
%%% End: 

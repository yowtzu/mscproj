\chapter{Evaluation and conclusions}
\graphicspath{{Chapter5/figures/}}
\label{EvaluationAndConclusion}
The work reported in the previous chapter provides evidence to support
the thesis hypothesis:
\begin{quote}
Sequential Monte Carlo (SMCs) have the potential to be an effective means of searching the optimal strategy for index tracking funds.
\end{quote}
This chapter reviews the work that has been done, evaluates the extent
to which they justify the thesis hypothesis and concludes the thesis
by addressing the directions for future work.

\subsection{Thesis contributions}
In the first experiment, we demonstrated the potential of SMCs in tracking a determinstic simple oscillating wave. The sensitivity of the techniques in terms of parameter settings and trade-off between the estimation accuracy and computational efforts are explored numerically. Several results are found. Firstly, it is found that a simple resampling step at each iteration performs better the more advanced method in which the resampling step is selectively triggered depending on the $ESS$ value. Secondly, it is found that the improvement of resample-move step is marginal. Considering the amount of computational cost involved and the extra amount of design cost involved, this may not be worthwhile. Thirdly, it is found that setting $\gamma$ as an increasing function of $t$ does lead to better performance, albeit extra care must be paid on the amount of increment at each step to avoid pre-mature convergence to local optimal. 

In the second experiment, we explore the potential of SMC techniques in searching the optimal strategy for index tracking error with constraints and minimizing the trasaction costs. We adopt the same framework with slightly different model. Although this framework has been applied in various engineering domains, there is no previous work to my knowledge in the application of such techniques in portfolio tracking error optimisation. In particular, we show how index can be fully replicated or partially replicated with transaction cost taken into account. The experiment results show the SMC techniques are rather promising.

In the third experiment, we introduce the concept of model predictive control (MPC) and how it can be used together with SMC techniques in improving tracking performance. Whist this technique is nothing new in control theory, its application to this portfolio optimisation problem is novel. The experimental results do suggest that MPC indeed improve the tracking performance.

\section{Envisaged future work}
Having discussed the contributions of the thesis, we now outline
numerous possible directions for future work that have been identified
during the course of this research.

\subsection{More realistic models}
The Arithmetic Brownian Model with drift used in this thesis is rather simple. A possible extension work is to consider a more advanced model, e.g., Geometric Brownian Model, Jump Diffusion model, etc. Moving away from conditional Gaussian model also introduces another complication. Without the conditional Gaussian assumption, the inner Kalman Filter recursion is no longer optimal. A possible solution to this is substituting the Kalman Filter with a nested SMC algorithm. This setup is known as the SMC2 algorithm \cite{CN13}.

\subsection{Parallel computation}
The nested SMC setup inevitably add consideration computation requirements. A possible speed up is to parallelise the steps in SMC algorithm. This is very straight-forward for all the steps, except the resampling step, which remains an interesting research topic on its own. 

\subsection{More complex financial indices}
The benchmark index used in this thesis is rather simple. This can
be potentially an issue. Having said that, this index is a simple, but by no means a ``toy''
index. It is a major financial index in the world. There are however indices with much large number of components, e.g., $\approx 1600$ for MSCI World Index. This translates to a high dimensional problem in wich may be difficult for the proposed model here.

A possible way to cope with this is to just track the index with partial replication using preselected subset of components, perhaps using Principle Component Analysis (PCA). Altenatively, we could use a divide and conquer approach, tracking multiple sub-indices separately. Yet, there is still much to answer here, for example:
\begin{itemize}
\item How to split an index into smaller components in a systematic manner?
\item How to deal with index components changes during annoucement and implementation?
\end{itemize}

\section{Closing remarks}
\label{ClosingRemark}
The work reported in this thesis demonstrates a considerable degree of
originality supported by extensive experimentation. The case studies
are necessarily limited given the limited amount of time frame.  However, the results
 demonstrate that portfolio optimisation using Sequential Monte Carlo techniques have very considerable promise. We recommend these approaches to
the research community for further investigation.


%%% ----------------------------------------------------------------------

% ------------------------------------------------------------------------

%%% Local Variables: 
%%% mode: latex

%%% TeX-master: "../thesis"
%%% End: 

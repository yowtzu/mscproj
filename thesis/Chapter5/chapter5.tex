\chapter{Conclusions and Future Work}
\graphicspath{{Chapter5/figures/}}
\label{cha:conclusions}
This thesis demonstrates how SMC can be used as an effective means of searching the optimal strategy for index tracking funds in terms of tracking error and transaction cost. The sensitivity of the technique with respect to the parameter settings and annealing techniques are explored. Several numerical examples using real-world data have been presented, showing how an index can be fully replicated or partially replicated with transaction cost taken into account. It also presents the Model Predictive Control (MPC) framework and demonstrates how the SMC technique can be integrated into MPC framework. The experimental results show the technique is very promising.

There are numerous possible directions for future work that have been identified during the course of this research.
\begin{enumerate}
\item More realistic models --- The Arithmetic Brownian Model with drift used in this thesis is rather simple. A possible extension work is to consider a more advanced model, e.g., Geometric Brownian Model, Jump Diffusion model, etc. Moving away from conditional Gaussian model is also possible. However, the inner Kalman Filter recursion is no longer optimal outsides the Gaussian assumption. A possible solution to this is substituting the Kalman Filter with a nested SMC algorithm. This setup is known as the SMC2 algorithm \cite{CN13}.

\item Parallel computation --- The nested SMC setup inevitably adds consideration amount of computation requirements. A possible way to speed-up is to parallelise some of the steps in SMC algorithm. This is very straight-forward for many steps, except the resampling step, which remains an interesting research topic on its own. 

\item More complex financial indices --- The benchmark index used in this thesis is rather simple. This can
be potentially an issue. Having said that, this index is a simple, but by no means a ``toy''
index. It is a major financial index in the world. There are however indices with much large number of components, e.g., $\approx 1600$ for MSCI World Index. This translates to a high dimensional problem in which may be difficult for the proposed model here. A possible way to cope with this is to just track the index with partial replication using pre-selected subset of components, perhaps using Principle Component Analysis (PCA). Alternatively, we could use a divide and conquer approach, tracking multiple sub-indices separately. Yet, there is still much to answer here, for example:
\begin{itemize}
\item How to split an index into smaller components in a systematic manner?
\item How to deal with index components changes during announcement and implementation?
\end{itemize}
\end{enumerate}
\section{Closing remarks}
\label{ClosingRemark}
The work reported in this thesis demonstrates a considerable degree of
originality supported by extensive experimentation. The case studies
are necessarily limited given the limited amount of time frame.  However, the results
 demonstrate that portfolio optimisation using Sequential Monte Carlo techniques have very considerable promise. We recommend these approaches to
the research community for further investigation.


%%% ----------------------------------------------------------------------

% ------------------------------------------------------------------------

%%% Local Variables: 
%%% mode: latex

%%% TeX-master: "../thesis"
%%% End: 

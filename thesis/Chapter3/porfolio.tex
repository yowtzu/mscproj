\chapter{Mean-variance portfolio}
Mean–variance portfolio selection is concerned with the allocation of wealth among a variety of securities so as to achieve the optimal trade-off between the expected return of the investment and its risk over a fixed planning horizon. The model was first proposed and solved more than fifty years ago in the single-period setting by Markowitz in his Nobel-Prize winning work Markowitz [41], [42]. With the risk of a portfolio measured by the variance of its return, Markowitz showed how to formulate the problem of minimizing a portfolio’s variance subject to the constraint that its expected return equals a prescribed level as a quadratic program. Such an optimal portfolio is said to be variance minimizing, and if it also achieves the maximum expected return among all portfolios having the same variance of return, then it is said to be efficient. The set of all points in the two-dimensional plane of variance (or standard deviation) and expected return that are produced by efficient portfolios is called the efficient frontier. Hence investors should focus on the efficient frontier, with different investors selecting different efficient portfolios, depending upon their risk preferences.
18

Not only does this model and its single-period variations (e.g., there might be con- straints on the investments in individual assets) witness widespread use in the financial industry, but also the basic concepts underlying this model have become the cornerstone of classical financial theory. For example, in Markowitz’s world (i.e., the world where all the investors act in accordance with the single-period, mean–variance theory), one of the important consequences is the so-called mutual fund theorem, which asserts that two mu- tual funds, both of which are efficient portfolios, can be established so that all investors will be content to divide their assets between these two funds. Moreover, if a risk-free asset (such as a bank account) is available, then it can serve as one of the two mutual funds. A logical consequence of this is that the other mutual fund, which itself is efficient, must correspond to the “market.” This, in turn, leads to the elegant capital asset pricing model (CAPM), see Sharpe [61], Lintner [37], Mossin [48].

Meanwhile, in subsequent years there has been considerable development of multi- period and, pioneered by the famous work Merton [45], continuous-time models for port- folio management. In these work, however, the approach is considerably different, as expected utility criteria are employed. For example, for the problem of maximizing the expected utility of the investor’s wealth at a fixed planning horizon, Merton [45] used dynamic programming and partial differential equation theory to derive and analyze the relevant Hamilton–Jacobi–Bellman (HJB) equation. The recent books by Karatzas and Shreve [26] and Korn [29] summarize much of this continuous time, portfolio management theory.
Multi-period, discrete-time mean–variance portfolio selection has been studied by Samuel- son [57], Hakansson [19], Grauer and Hakansson [18], and Pliska [54]. But somewhat surprisingly, the exact, faithful continuous-time versions of the mean–variance problem have not been developed until very recently. This is surprising because the mean–variance portfolio problem is known to be very similar to the problem of maximizing the expected quadratic utility of terminal wealth. Solving the expected quadratic utility problem can produce a point on the mean–variance efficient frontier, although a priori it is often unclear what the portfolio’s expected return will turn out to be. So while it is straightforward
19
to formulate a continuous-time version of the mean–variance problem as a dynamic pro- gramming problem, researchers have been slow to produce significant results.

A more modern approach to continuous-time portfolio management, first introduced by Pliska [52],[53], avoids dynamic programming by using the risk neutral (martingale) probability measure; but this has not been much helpful either. This risk neutral com- putational approach decomposes the problem into two sub-problems, where first one uses convex optimization theory to find the random variable representing the optimal terminal wealth, then solves the sub-problem of finding the trading strategy that replicates the terminal wealth. The solution for the mean–variance problem of the first sub-problem is known for the unconstrained case, 1 but apparently nobody has successfully solved for continuous time applications the second sub-problem, which is essentially a martingale representation problem.
A breakthrough of sorts was provided in a recent paper by Li and Ng [33], who studied the discrete-time, multi-period, mean–variance problem using the framework of multi- objective optimization, where the variance of the terminal wealth and its expectation are viewed as competing objectives. They are combined in a particular way to give a single- objective “cost” for the problem. An important feature of this paper is an embedding technique, introduced because dynamic programming could not be directly used to deal with their particular cost functional. Their embedding technique was used to transform their problem to one where dynamic programming was used to obtain explicit optimal solutions.

Zhou and Li [70] used the embedding technique and linear–quadratic (LQ) optimal control theory to solve the continuous-time, mean–variance problem with assets having deterministic diffusion coefficients. In their LQ formulation, the dollar amounts, rather than the proportions of wealth, in individual assets are used to define the trading strategy. This leads to a dynamic system that is linear in both the state (i.e., the level of wealth) and the control (i.e., the trading strategy) variables. Together with the quadratic form of the objective function, this formulation falls naturally into the realm of stochastic LQ control. Moreover, since there is no running cost in the objective function, the resulting problem is inherently an indefinite stochastic LQ control problem, the theory of which has been developed only very recently (see, e.g., [67, Chapter 6]).
1See, for example, Pliska [54]; the treatment there was for the single-period situation, but the basic result easily generalizes to very similar results for the multi-period and continuous-time situations.


Exploiting the stochastic LQ control theory, Zhou and his colleagues have considerably extended the initial continuous-time, mean–variance results obtained by Zhou and Li [70]. Lim and Zhou [36] allowed for stocks which are modeled by processes having random drift and diffusion coefficients, Zhou and Yin [71] featured assets in a regime switching market, and Li, Zhou, and Lim [34] introduced a constraint on short selling. Kohlmann and Zhou [27] went in a slightly different direction, studying the problem of mean–variance hedging of a given contingent claim. In all these papers, explicit forms of efficient/optimal portfolios and efficient frontiers were presented. While many results in the continuous-time Markowitz world are analogous to their single-period counterparts, there are some results that are strikingly different. Most of these results are summarized by Zhou [69], who also provided a number of examples that illustrate the similarities as well as differences between the continuous-time and single-period settings.
In view of all this recent work on the continuous-time, mean–variance problem, what is left to be done? The answer is that it is desirable to address a significant shortcoming of the preceding models, for their resulting optimal trading strategies can cause bankruptcy for the investor. Moreover, these models assume a bankrupt investor can keep on trading, borrowing money even though his or her wealth is negative. In most of the portfolio optimization literature the trading strategies are expressed as the proportions of wealth in the individual assets, so with technical assumptions (such as finiteness of the integration of a portfolio) about these strategies the portfolio’s monetary value will automatically be strictly positive. But with strategies described by the money invested in individual assets, as dictated by the stochastic LQ control theory approach, a larger set of trading strategies is available, including ones which allow the portfolio’s value to reach zero or to become and remain strictly negative (e.g., borrow from the bank, buy stock on margin, and watch the stock’s price go into the tank). The ability to continue trading even though the value of an investor’s portfolio is strictly negative is highly unrealistic. This brings us to the subject of this chapter: the study of the continuous-time, mean–variance problem with the additional restriction that bankruptcy is prohibited2.
2Here the bankruptcy is defined as the wealth being strictly negative. A zero wealth is not regarded 21

Chapter 3 Mean-Variance Criteria in a Complete Market
In this chapter we use an extension of the risk neutral approach rather than making heavy use of stochastic LQ control theory. However, we retain the specification of trading strategies in terms of the monetary amounts invested in individual assets, and we add the explicit constraint that feasible strategies must be such that the corresponding monetary value of the portfolio is nonnegative (rather than strictly positive) at every point in time with probability one. The resulting continuous time, mean–variance portfolio selection problem is straightforward to formulate, as will be seen in the following section. Our model of the securities market is complete, although we allow the asset drift and diffusion coefficients, as well as the interest rate for the bank account, to be random. Once again, we emphasize that the set of trading strategies we consider is larger than that of the proportional strategies, and we will show that the efficient strategies we obtain are in general not obtainable by the proportional ones. In Section 3.2 we also demonstrate that the original nonnegativity constraint can be replaced by the constraint which simply requires the terminal monetary value of the portfolio to be nonnegative. This leads to the first sub-problem in the risk neutral computational approach: find the nonnegative random variable having minimum variance and satisfying two constraints, one calling for the expectation of this random variable under the original probability measure to equal a specified value, and the other calling for the expectation of the discounted value of this random variable under the risk neutral measure to equal the initial wealth.
In Section 3.3 we study the feasibility of our problem, an issue that has never been
addressed by other authors to the best of our knowledge. There we provide two nonnegative
numbers with the property that the variance minimizing problem has a unique, optimal
solution if and only if the ratio of the initial wealth to the desired expected wealth falls
between these two numbers. In Section 3.4 we solve the first sub-problem by introducing
two Lagrange multipliers that enable the problem to be transformed to one where the only
constraint is that the random variable, i.e., the terminal wealth, must be nonnegative.
This leads to an explicit expression for the optimal random variable, an expression that
is in terms of the two Lagrange multipliers which must, in turn, satisfy a system of
two equations. In Section 3.5 we show that this system has a unique solution, and we
as in bankruptcy. In fact, as will be seen in the sequel the wealth process associated with an efficient portfolio may indeed “touch” zero with a positive probability.
22

Chapter 3 Mean-Variance Criteria in a Complete Market
establish simple conditions for determining what the signs of the Lagrange multipliers. A consequence here is the observation that the optimal terminal wealth can be interpreted as the payoff of either, depending on the signs of the Lagrange multipliers, a European put or a call on a fictitious security.
In Section 3.6 we turn to the second sub-problem, showing that the optimal trading strategy of the variance minimizing problem can be expressed in terms of the solution of a backward stochastic differential equation. We also provide an explicit characterization of the mean–variance efficient frontier, which is a proper portion of the variance minimizing frontier. Unlike the situation where bankruptcy is allowed, the expected wealth on the efficient frontier is not necessarily a linear function of the standard deviation of the wealth. In Section 3.7 we consider the special case where the interest rate and the risk premium are deterministic functions of time (if not constants). Here we provide explicit expressions for the Lagrange multipliers, the optimal trading strategies, and the efficient frontier. We conclude in Section 3.8 with some remarks.
Somewhat related to our work are the continuous-time studies of mean–variance hedg- ing by Duffie and Richardson [12], and Schweizer [59]. More pertinent is the study of continuous-time, mean–variance portfolio selection in Richardson [55], a study where the portfolio’s monetary value was allowed to become strictly negative. Also in the working paper of Zhao and Ziemba [68], a mean–variance portfolio selection problem with deter- ministic market coefficients and with bankruptcy allowed is solved using a martingale approach. Closely connected to our research is the work by Korn and Trautmann [30] and Korn [29]. They considered the continuous-time mean–variance portfolio selection problem with nonnegativity constraints on the terminal wealth for the case of the Black– Scholes market where there is a single risky asset that is modelled as simple geometric Brownian motion and where the bank account has a constant interest rate. They provided expressions for the optimal terminal wealth as well as the optimal trading strategy using a duality method. Their first sub-problem fixes a single Lagrange multiplier and then solves an unconstrained convex optimization problem for the optimal proportional strat- egy. Their second sub-problem is to find the “correct” value of their Lagrange multiplier. Actually, they do not have an explicit constraint for nonnegative wealth, but by using
23
or, equivalently,
Then we have
ρ(0) = 1, ôôt1ôtô
ρ(t) = exp − [r(s) + 2 |θ(s)|2 ]ds − θ(s)′ dW (s) . 00
(3.3)
(3.4)
Chapter 3 Mean-Variance Criteria in a Complete Market
strategies that are in terms of proportions of wealth, a strictly positive wealth is automat- ically achieved. In our paper we include strategies that allow the wealth to become zero at intermediate dates, so apparently our set of feasible strategies is larger. Our results are considerably more general, for we allow stochastic interest rates, an arbitrary num- ber of assets, and asset drift and diffusion coefficients that are random. And we provide characterizations of efficient frontiers, necessary and sufficient conditions for existence of solutions, and several other kinds of results that Korn and Trautmann [30] did not address at all.
3.2 Problem formulation
We adopt the market model given in Chapter 2. In this chapter, we suppose the number of risk securities, n, is equal to m, the dimension of the Brownian motion. In addition, we make the following basic assumption throughout this chapter
Assumption 3.2.1
σ(t)σ(t)′ ≥ δIm, ∀t ∈ [0, T ], a.s., (3.1) for some δ > 0, where Im is the m × m identity matrix.
By Theorem 2.3.1 and Theorem 2.4.1, the market is arbitrage free and complete. Fur- thermore, there exist a uniformly bounded θ(·) = σ(t)−1B(t). Note that θ(·) is the only process satisfying σ(·)θ(·) = B(·), which is called the risk premium process.
As in Chapter 2, define ρ(·) to be the deflator process as
 dρ(t)=ρ(t)[−r(t)dt−θ(t)′dW(t)],
(3.2)
x(t) = ρ(t)−1E(ρ(T )x(T )|Ft), ∀t ∈ [0, T ] for any wealth process x(·).
24
Chapter 3 Mean-Variance Criteria in a Complete Market
With (3.4), the wealth process x(·) is nonnegative if and only if the terminal wealth x(T ) is nonnegative. From the economic standpoint, this is a consequence of the fact that there exists a risk neutral probability measure under which the discounted wealth process is a martingale. Hence if the terminal wealth is nonnegative, then so is the discounted wealth process and thus x(·). This property can help us greatly simplify our problem, which we formulate a little later.
It should be emphasized an important point concerning the way we specify our trading strategies. Most papers in the research literature define a trading strategy or portfolio, say u(·), as the (vector of) proportions or fractions of wealth allocated to different assets, perhaps with some other “technical” constraints such as ô0T |u(t)|2dt < ∞, a.s., being specified (see, e.g., Cvitanic and Karatzas (1992) and Karatzas and Shreve (1998)). With this definition, and if additionally the self-financing property is postulated, then the wealth at any time t ≥ 0 can be shown to be proportional to the wealth at time t = 0, in the sense that x(t) = x0x ̃(t), where x ̃(t) is an (almost surely) strictly positive process. In fact, with a proportional, self-financing strategy u(·) satisfying the above condition, it can be shown that the wealth process is a unique strong solution of the following equation

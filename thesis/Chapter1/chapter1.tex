\chapter{Introduction}
\graphicspath{{Chapter1/figures/}}
\label{Introduction}
Capital allocation is a common challenge for every investor. In the investment decision making process, investors decide how to allocate capital to different investable assets to form a portfolio that performs better than any other possible portfolios according to some criteria. These criteria can be very different among the investors. Some investors may also have additional considerations such as tax considerations and legal restrictions on investment assets or holding periods.

Two common objectives, which often contradicting, are the financial return and the investment risk. The Markowitz's modern portfolio theory \cite{HM52} proposes a portfolio selection framework in which it is assumed that investors are interested in maximising their portfolio's return and minimising their portfolio's financial risk (measured by the variance of the portfolio's return). Based on these criteria, the set of non-dominated portfolios, commonly known as the \emph{efficient portfolios}, of a given investment period can be found. However, using the variance of the portfolio's return as the risk measure has its limitation. Variance is a symmetric measure; an out-performing asset is deemed to as risky as an under-performing one. Many alternative risk measurements have been proposed, e.g. Sortino ratio, Conditional Value at Risk (CVaR), etc.; see \cite{RTR00} for details.

There are some investment fund managers who have no strong interest in maximising their portfolio's return. Instead, the main objective of portfolio management for such a fund is to simply track and replicate the exposure of a benchmark index as close as possible. These funds are attractive as they provide the investors the exposure to the market at low fees and taxes because only minimal active management is required. Passively tracking a benchmark index also makes the fund less vulnerable to the change of fund managers. The performance of these funds are often assessed in term of how well the fund tracks the benchmark index using some pre-defined metrics. These index tracking funds are the application of interest in this thesis.

\section{Methodology}
In the era before modern computing, portfolio optimisation have been explored in an analytical fashion, adopting necessary assumption as necessary. This seems rather restrictive; there are many instances where numerical methods have been used to derive an approximate or even more effective solution to the problem in question. For example, Monte Carlo methods are used to approximate integral, heuristic optimisation search techniques such as simulated annealing applied in engineering domains, etc.

In this thesis, we view a portfolio optimisation as a \emph{stochastic} control problem. We adopt the Bayesian view and treat the controls as random variables. The objective is to find the sequence of control parameters that optimises the reward function defined in terms of the tracking error between a fund and its benchmark index. We investigate the potential of using Sequential Monte Carlo (SMC) as the means of determining the optimal strategies for this problem. The main reason of choosing SMC is its ability to carry out \emph{sequential} update on the posterior distribution over time fit well with the problem in question and its success in its applications in many different domains. Of course, other heuristic search techniques are also potentially applicable.

To investigate this approach, we first used SMC to track the output of a simple deterministic reference model. This model is doubly useful. It demonstrates the concept nicely and serves as a basic model to allow us to gain further understanding on the tuneable parameters. We then considered the problem of tracking a real-world index with its constituent prices modelled as Arithmetic Brownian motion with drift. Using SMC, we search for the optimal strategy (the optimal values of the control parameters at each time point) with respect to the reward function. The reward function is defined in terms of the tracking error and the transaction costs involved in the changes of the portfolio's holding position. Lastly, we introduce the Model Predictive Control (MPC) framework and show how to integrate this technique into the MPC framework easily.

\section{Contributions}
In this thesis, it is found that SMC has the potential to be an effective means of searching the optimal strategy for index tracking funds. The following work has been carried out:
\begin{enumerate}
\item exploring the potential of SMC in searching the optimal strategy for minimising the tracking error and transaction costs of an index fund. 
\item exploring the sensitivity of SMC in terms of the parameter settings, the trade-off between estimation accuracy and computational efforts numerically and providing suggestions for real-world problem.
\item introducing the Model Predictive Control (MPC) framework and demonstrate how to integrate SMC technique proposed into the MPC framework.
\end{enumerate}

Given the time frame of the project, we fully understand it is impossible to evaluate our approach on full scale strategy. The aim is to establish the plausibility or, at the very least, a greater understanding of the strengths and weaknesses of the above approach.

\section{Thesis organisation}
The subsequent chapters of this thesis are organised as follows:
\begin{itemize}
\item Chapter \ref{cha:mcmethods} reviews some fundamental concepts in Monte Carlo methods that are related to this thesis. It begins with a brief introduction to basic sampling methods such as perfect Monte Carlo sampling, rejection sampling and importance sampling. It then presents Sequential Monte Carlo (SMC) methods include Sequential Importance Sampling (SIS), Sequential Importance Resampling (SIR) and various further enhancements proposed in the literature, e.g., Effective Sample Size (ESS), Resample-Move step, Rao-Blackwellisation, that are being used in this thesis.
\item Chapter \ref{cha:po} briefly reviews the portfolio optimisation problem in practice. It then discusses how a portfolio problem can be transformed into a path-space parameter estimation problem. Lastly, it presents a simple experiment in which we attempt to use SMC to track a reference signal generated by a deterministic reference model.
\item Chapter \ref{cha:dax} details the application of the SMC technique to track the DAX index. It presents several experiments that have been carried out to demonstrate how the index can be tracked with full replication as well as partial replication of its constituents. Lastly, it introduces Model Predictive Control (MPC) framework and demonstrates how this technique can be integrated into the MPC framework easily.
\item Chapter \ref{cha:conclusions} concludes the thesis by evaluating the contributions of the thesis and discusses some potential work for the future.
\end{itemize}

%%% ----------------------------------------------------------------------


%%% Local Variables: 
%%% mode: latex
%%% TeX-master: "../thesis"
%%% End: 

\chapter{Introduction}
\graphicspath{{Chapter1/figures/}}
\label{Introduction}
Capital allocation is a common challenge for every investor. In the investment decision making process, investors decide how much capital is allocated to different investable assets to form a portfolio that performs better than any other possible portfolios according to some criteria. These criterion can be different among investors. Some investors may also have additional considerations such as tax considerations and legal restriction on investment assets or holding periods.

Two common objectives, which often contradicting, are the financial return and the investment risk. The Markowitz's modern portfolio theory \cite{HM52} proposes a portfolio selection framework. In the Markowitz's framework, it is assumed that investor attempt to maximize a portfolio's return and minimize the risk (measured by the variance of the portfolio return). Based on these criteria, the set of non-dominated portfolios, commonly known as the \emph{efficient portfolios} of a given investment period can be found. However, using variance as a risk measure has its limitation. Variance is a symmetric measure; an out-performing asset (than the expected return) is deemed to as risky as an under-performing one. Many alternative risk measurements have been proposed, e.g. Sortino ratio, Conditional Value at Risk (CVaR), etc.; see \cite{RTR00} for details.

There are some investment managers who have no interest in maximising their portfolio's return. Instead, the main objective of portfolio management for such a fund is to simply track and replicate the exposure of a benchmark index as close as possible. These funds are attractive as it provides the investors the exposure to the market at low fees and taxes because only minimal active management is required. Passively tracking a benchmark index also makes the fund less vulnerable to the change of fund managers. The performance of these funds are often assessed in term of how well the fund tracks the benchmark index using some pre-defined metrics. These index tracking funds are the application of interest in this thesis.

\section{Metholodogy}
In the era before modern computing, portfolio optimization have been explored in an analytical fashion, adopting necessary assumption as necessary. This seems rather restrictive; there are many instances where numerical methods have been used to derive an approximate or even more effective solution to the problem in question. For example, Monte Carlo technique is used to approximate integral, heuristic optimisation search techniques such as simulated annealing applied in engineering domains, etc.

In this thesis, we view a portfolio optimisation as a \emph{stochastic} control problem. We adopt the Bayesian view and treat the controls as random variables. The objective is to find the sequence of control parameters that optimise the reward function defined in terms tracking error between the fund and the benchmark index. We investigate the potential of using Sequential Monte Carlo (SMC) as the means of determining the optimal strategies for the portfolio optimisation problem in question. The main reason of choosing SMC is its ability to carry out \emph{sequential} update on the posterior distribution over time fit well with the problem in question and its success in its applications in many different domains. Of course, other heuristic search techniques are also potentially applicable.

To investigate this approach, we first applied the technique on to track the output of a simple deterministic reference model. This model is doubly useful. It demonstrates the concept nicely and serves as a basic model to allow us to gain further understanding on the tunable parameters. We then considered the problem of tracking a real-world index with its constituent prices modelled as Brownian motion with drift. Using SMC, we search for the optimal strategy (the set of control parameters at each time point) that optimise against the reward function defined in terms of minimizing the tracking error and the transaction costs involved in maintaining the positions. Lastly, we introduce the concept of Model Predictive Control (MPC) and this concept can be used here to improve the tracking performance.

\section{Contributions}
In this thesis, it is found that SMC has the potential to be an effective means of searching the optimal strategy for index tracking funds. We have:
\begin{enumerate}
\item explored the potential of SMC techniques in searching the optimal strategy from index tracking error with constraints and minimizing the transaction costs.
\item explored the sensitivity of the techniques in terms of the parameter settings, trade-off between the estimation accuracy and computational efforts numerically and providing suggestions for real-world problem.
\item introduced the concept of model predictive control (MPC) and how it can be used together with SMC in improving tracking performance.
\end{enumerate}

Given the time frame of the project, we fully understand it is impossible to evaluate our approach on full scale strategy. The aim is to establish the plausibility or, at the very least, a greater understanding of the strengths and weaknesses of the above approach.

\section{Thesis organisation}
The subsequent chapters of this thesis are organised as follows:
\begin{itemize}
\item Chapter 2 reviews some fundamental concepts in Monte Carlo methods that are related to this thesis. It begins with a brief introduction to basic methods such as perfect Monte Carlo sampling, rejection sampling, importance sampling. It then  introduces the Sequential Monte Carlo (SMC) techniques, along with various enhancements proposed, e.g., MCMC move, Marginalisation, used in this thesis.
\item Chapter 3 briefly review the some literature on the portfolio optimization problem. It then discusses how a portfolio problem can be transformed naturally into a path-space parameter estimation problem. It then presents a toy experiment in which we attempt to use SMC to track a reference signal generated by a known synthetic model. Next, it presents experiment in using SMC to infer the optimal control for portfolio that tracks real-world indices. In particular, we focus on the major stock indices, how we can track the index using SMC techniques. Lastly, it introduces the MPC techniques to further improve the tracking performance.
\item Chapter 4 concludes the thesis by evaluating the contributions of the thesis and discusses some potential work for the future.
\end{itemize}

%%% ----------------------------------------------------------------------


%%% Local Variables: 
%%% mode: latex
%%% TeX-master: "../thesis"
%%% End: 
